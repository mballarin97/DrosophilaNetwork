\section{Appendix}
The figures related to the report are displayed here.

% COARSE GRAINING METRICS EVOLUTION
\begin{figure}[h]
    \centering
    \includegraphics[width=\textwidth]{Images/Coarse/Evolution_graph_coearse.png}
    \caption{Evolution of different metrics along the coarse graining procedure.
    We notice an outlier at $3000$ neurons, that we will analyze in details. We understand
    the various metrics does not change too much up to $10000$ neurons, which is half 
    of the system.}
    \label{fig:coarse_evol}
\end{figure}

% OUTLIER DISTRIBUTIONS
\begin{figure}
    \centering
    \includegraphics[width=\textwidth]{Images/Coarse/Outlier_distribution.png}
    \caption{Degree and degree connectivity distribution for the outlier point at $N'=3000$ neurons.}
    \label{fig:n3000}
\end{figure}

% DEGREE CONNECTIVITY DISTRIBUTION
\begin{figure}
    \centering
    \includegraphics[width=\textwidth]{Images/Coarse/Avg_deg_con_distr.png}
    \caption{Average degree connectivity distribution for different number of neurons along
    the clustering coarse graining technique.}
    \label{fig:con_distr}
\end{figure}

% RANDOM ATTACK
\begin{figure}
	\centering
	\begin{subfigure}[b]{0.45\textwidth}
		\centering
		\includegraphics[width=\textwidth]{Images/plots_rnd/rnd_20.pdf}
		\caption{for $10500$ neurons}
		%\label{fig:y equals x}
	\end{subfigure}
	\hfill
	\begin{subfigure}[b]{0.45\textwidth}
		\centering
		\includegraphics[width=\textwidth]{Images/plots_rnd/rnd_22.pdf}
		\caption{for $9500$ neurons}
		%\label{fig:three sin x}
	\end{subfigure}
	\\ \vspace{5mm}
	\begin{subfigure}[b]{0.45\textwidth}
		\centering
		\includegraphics[width=\textwidth]{Images/plots_rnd/rnd_28.pdf}
		\caption{for $6500$ neurons}
		%\label{fig:y equals x}
	\end{subfigure}
	\hfill
	\begin{subfigure}[b]{0.45\textwidth}
		\centering
		\includegraphics[width=\textwidth]{Images/plots_rnd/rnd_34.pdf}
		\caption{for $3500$ neurons}
		%\label{fig:three sin x}
	\end{subfigure}
	\\ \vspace{5mm}
	\begin{subfigure}[b]{0.45\textwidth}
		\centering
		\includegraphics[width=\textwidth]{Images/plots_rnd/rnd_38.pdf}
		\caption{for $1500$ neurons}
		%\label{fig:y equals x}
	\end{subfigure}
	\hfill
	\begin{subfigure}[b]{0.45\textwidth}
		\centering
		\includegraphics[width=\textwidth]{Images/plots_rnd/rnd_40.pdf}
		\caption{for $500$ neurons}
		%\label{fig:three sin x}
	\end{subfigure}
	\\ \vspace{5mm}
	

	\caption{Robustness of the network at various graining scales against random node removal attacks, using the largest connected component as measure.}
	\label{fig:rnd_atk}
\end{figure}


% BETWEENNESS ATTACK
\begin{figure}
	\centering
	\begin{subfigure}[b]{0.45\textwidth}
		\centering
		\includegraphics[width=\textwidth]{Images/plots_ib/ib_20.pdf}
		\caption{for $10500$ neurons}
		%\label{fig:y equals x}
	\end{subfigure}
	\hfill
	\begin{subfigure}[b]{0.45\textwidth}
		\centering
		\includegraphics[width=\textwidth]{Images/plots_ib/ib_22.pdf}
		\caption{for $9500$ neurons}
		%\label{fig:three sin x}
	\end{subfigure}
	\\ \vspace{5mm}
	\begin{subfigure}[b]{0.45\textwidth}
		\centering
		\includegraphics[width=\textwidth]{Images/plots_ib/ib_28.pdf}
		\caption{for $6500$ neurons}
		%\label{fig:y equals x}
	\end{subfigure}
	\hfill
	\begin{subfigure}[b]{0.45\textwidth}
		\centering
		\includegraphics[width=\textwidth]{Images/plots_ib/ib_34.pdf}
		\caption{for $3500$ neurons}
		%\label{fig:three sin x}
	\end{subfigure}
	\\ \vspace{5mm}
	\begin{subfigure}[b]{0.45\textwidth}
		\centering
		\includegraphics[width=\textwidth]{Images/plots_ib/ib_38.pdf}
		\caption{for $1500$ neurons}
		%\label{fig:y equals x}
	\end{subfigure}
	\hfill
	\begin{subfigure}[b]{0.45\textwidth}
		\centering
		\includegraphics[width=\textwidth]{Images/plots_ib/ib_40.pdf}
		\caption{for $500$ neurons}
		%\label{fig:three sin x}
	\end{subfigure}
	\\ \vspace{5mm}
	
	
	\caption{Robustness of the network at various graining scales against removal of highest-betweenness nodes attack, using the largest connected component as measure.}
	\label{fig:ib_atk}
\end{figure}


% CASCADING ATTACK
\begin{figure}
	\centering
	\begin{subfigure}[b]{0.45\textwidth}
		\centering
		\includegraphics[width=\textwidth]{Images/plots_cascading/cascading_20.pdf}
		\caption{for $10500$ neurons}
		%\label{fig:y equals x}
	\end{subfigure}
	\hfill
	\begin{subfigure}[b]{0.45\textwidth}
		\centering
		\includegraphics[width=\textwidth]{Images/plots_cascading/cascading_22.pdf}
		\caption{for $9500$ neurons}
		%\label{fig:three sin x}
	\end{subfigure}
	\\ \vspace{5mm}
	\begin{subfigure}[b]{0.45\textwidth}
		\centering
		\includegraphics[width=\textwidth]{Images/plots_cascading/cascading_28.pdf}
		\caption{for $6500$ neurons}
		%\label{fig:y equals x}
	\end{subfigure}
	\hfill
	\begin{subfigure}[b]{0.45\textwidth}
		\centering
		\includegraphics[width=\textwidth]{Images/plots_cascading/cascading_34.pdf}
		\caption{for $3500$ neurons}
		%\label{fig:three sin x}
	\end{subfigure}
	\\ \vspace{5mm}
	\begin{subfigure}[b]{0.45\textwidth}
		\centering
		\includegraphics[width=\textwidth]{Images/plots_cascading/cascading_38.pdf}
		\caption{for $1500$ neurons}
		%\label{fig:y equals x}
	\end{subfigure}
	\hfill
	\begin{subfigure}[b]{0.45\textwidth}
		\centering
		\includegraphics[width=\textwidth]{Images/plots_cascading/cascading_40.pdf}
		\caption{for $500$ neurons}
		%\label{fig:three sin x}
	\end{subfigure}
	\\ \vspace{5mm}
	
	
	\caption{Robustness of the network at various graining scales against removal of highest-betweenness nodes cascading attack, using the largest connected component as measure.}
	\label{fig:cascading_atk}
\end{figure}

% ADJACENCY MATRIX
\begin{figure}
    \centering
    \includegraphics[width=\textwidth]{Images/Communities/Adj_matrix.png}
    \caption{Non weighted adjacency matrix of the full system. The neurons are ordered by community
    and then displayed in a decreasing degree order. In the y-axis the zone is, starting from its label, 
    up to the lower label. Similarly, for the x-axis the zone starts from the label and finishes to the next
    label to the right.}
    \label{fig:adj}
\end{figure}

% COMMUNITY EVOLUTION
\begin{figure}
    \centering
    \begin{subfigure}[t]{0.49\textwidth}
		\centering
		\includegraphics[width=\textwidth]{Images/Communities/Communities_after_graining.png}
		\caption{Density of the synapses communities before the coarse graining in gray, and 
		after the coarse graining with colors. We notice that the densities strongly vary in the 
		procedure.}
	\end{subfigure}
	\hfill
	\begin{subfigure}[t]{0.49\textwidth}
		\centering
		\includegraphics[width=\textwidth]{Images/Communities/Density_neurons_21733.png}
		\caption{Density of neurons before the graining procedure. The distribution is similar,
		even if not the same, of the synapses distribution.}
	\end{subfigure}
	\caption{}
	\label{fig:com_evol}
\end{figure}

\begin{figure}
	\centering
	\begin{subfigure}[b]{0.49\textwidth}
		\centering
		\includegraphics[width=\textwidth]{Images/Communities/Error_comm_21733.png}
		\caption{$21733$ neurons}
	\end{subfigure}
	\hfill
	\begin{subfigure}[b]{0.49\textwidth}
		\centering
		\includegraphics[width=\textwidth]{Images/Communities/Error_comm_17000.png}
		\caption{$17000$ neurons}
	\end{subfigure}
	\\ \vspace{5mm}
	\centering
	\begin{subfigure}[b]{0.49\textwidth}
		\centering
		\includegraphics[width=\textwidth]{Images/Communities/Error_comm_12000.png}
		\caption{$12000$ neurons}
	\end{subfigure}
	\hfill
	\begin{subfigure}[b]{0.49\textwidth}
		\centering
		\includegraphics[width=\textwidth]{Images/Communities/Error_comm_7000.png}
		\caption{$7000$ neurons}
	\end{subfigure}
	\\ \vspace{5mm}
	\caption{Caption}
	\label{fig:cmaps}
\end{figure}