In this work we will analyze the hemibrain of a Drosophila, commonly called “small fruit fly”.
We will apply all the techniques proper of the network science field to a neural structural connectome,
thus connecting two fields that can really 
benefit from each others. One of the most challenging task of this project will be handling
the dataset, since it amounts of more than $20000$ neurons and $2000000$ synapses.
A technique to reduce the network size will be developed, while monitoring important metrics. 
We will analyze how the network robustness changes along the procedure.
Finally, we will focus on the neural communities, trying to understand 
if we are able to distinguish the macro-areas of the brain at different network sizes.