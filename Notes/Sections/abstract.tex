In this work we will analyze the hemibrane of o Drosophila, commonly called “small fruit fly”.
This analysis will be fascinating since we will apply all the techniques proper of the 
network science field to a neural connectome, thus connecting two fields that can really 
benefit of each others. One of the most challenging section of this project will be handling
the dataset, since it amounts of more than $2\cdot10^{4}$ neurons and $2\cdot10^6$ synapses.
We will so develop a technique to reduce the network size, trying to induce the smallest
possible variation to the important quantities. Finally, we will pose particular interest
in the analysis of the communities, trying to understand if we are able to distinguish
the different macro-areas of the brain.