\section{Conclusions \label{sec:concl}}
We can finally say that the coarse graining technique is a very
interesting procedure that let us reduce the size of a network, 
maintaining some important quantities. Indeed, it can even help in 
evaluating some of the metrics we used, showing new interesting
insight for the network. We wrap up our results, dividing them 
in the different section of the report.
\begin{itemize}
    \item The coarse graining procedure using the clustering technique
        is meaningful, producing interesting results even when we reach
        less than $1/10$ of the original size of the network;
    \item The coarse graining procedure seems able to preserve the 
        graph robustness properties. This is particularly useful, since
        it is really time consuming to run such simulations. This means
        that we can perform meaningful analysis on smaller graph and so
        deduce the effect on the larger ones, sparing a lot of resources;
    \item The information on the communities are conserved along the coarse
        graining, and the more connected communities become even more distinguishable
        after the application of the procedure. We also highlighted an interesting
        fact: the more distinguishable communities are the ones involved in 
        learning and memory.
\end{itemize} 
We can finally conclude that the technique developed is an interesting
tool, which can give new insights on the studied network.
It could be interesting to apply it to other networks, to see if this really
is a more general tool that can help also in other analysis.
