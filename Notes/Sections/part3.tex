\section{Communities \label{sec:comm}}
We analyze here the community detection task. As we already state we have the synapses location
as an entry of the dataset. However, certain synapses had different locations.
This is due to the fact that, even if different zones are more specialized 
in one task, the brain should be considered more as a global entity than as the union 
of different zones.
Nevertheless, we decided to select the simplest path possible, and assign to the synapse $i$ the community (location) $C_i$
if $C_i$ has the highest weight value. We present now the different zones of the 
drosophila brain: 
\begin{enumerate}
    \item \tb{AL}: antennal lobe.
        The antennal lobe is the primary olfactory brain area, and it is a sphere-shaped 
        deutocerebral neuropil in the brain that receives input from the olfactory sensory neurons 
        in the antennae and mouthparts;
    \item \tb{CX}: Central complex. It is the center that is assigned to
         navigation, sleep, learning, memory, nociception (perception of pain);
    \item \tb{GNG}:gnathal ganglia, devoted to taste and feeding;
    \item \tb{INP}: Inferior neuropils. It is an area in the nervous system composed of mostly unmyelinated axons, dendrites and glial 
        cell processes that forms a synaptically dense region containing a relatively low number 
        of cell bodies;     
    \item \tb{LH}: Lateral horn. It is one of the two areas of the insect brain where projection
         neurons of the antennal lobe send their axons. The other area is the mushroom body.
         Several morphological classes of neurons in the lateral horn receive olfactory 
         information through the projection neurons lateral horn, axons of 
         pheromone-sensitive projection neurons are segregated from the axons of 
         plant odor-sensitive projection neurons. In addition, the dendrites of lateral 
         horn neurons are restricted to one of these two zones, suggesting that pheromones 
         and plant odors are processed separately in the lateral horn.
    \item \tb{LX}: Lateral complex, similar to the central complex;
    \item \tb{MB}: Mushroom body. It is known to play a role in olfactory learning and memory. 
         The mushroom bodies and the lateral horn are the two higher brain regions that receive 
         olfactory information from the antennal lobe via projection neurons;
    \item \tb{OL}: Optical lobe. It sits behind the eye and is responsible for the processing
         of the visual information. It is made up of three layers;
    \item \tb{PENP}: periesophageal neuropils, which has the same properties 
        of the \tb{INP} zone;
    \item \tb{SNP}: Superior neuropils, which has the same properties 
        of the \tb{INP} zone;
    \item \tb{VLNP}: Ventrolateral neuropils, which has the same properties 
        of the \tb{INP} zone;
    \item \tb{VMNP}: ventromedial neuropils, which has the same properties 
        of the \tb{INP} zone;
    \item \tb{Not Primary}: other zones which are not clearly defined.
\end{enumerate}

In most cases what we need are the communities of nodes, not the one of edges. We so define a 
simple rule to translate the information from the edges to the nodes: 
$$
\mbox{Node $i$ is in the community $C_i$ if most of its synapses are in the community $C_i$}
$$
We show in Figure \ref{fig:adj} the non-weighted adjacency matrix of the full system, where we
first order the neurons by their community, and then, inside the community, in a decreasing 
degree order. We notice that, in general, nodes are not connected inside a community but are 
mostly connected to other zones. This is particularly true for the neuropils.

We want to see how the communities evolves under the coarse graining, and we so define a simple rule
to keep track of it. The synapses between the supernode $I$ and the supernode $J$ after an iteration
of the clustering coarse graining is in the community $C_l$ if most of the old synapses between
nodes in $I$ and $J$ belonged to the community $C_l$.

In Figure \ref{fig:com_evol} we show the synaptic and neuronal communities at the initial size and after the
coarse graining. We can understand, by recalling our method for the community graining, that the 
zones that loose density are more sparse, i.e. not a lot of the synapses of the same community connects
same areas, while the ones that increases are more strongly interconnected. We so understand that the zones
that are more strongly interconnected are the Mushroom body, the Central Complex and the lateral horn.
The superior neuropils also increases, but its relative increase is minimal, and by looking at the 
full time evolution we would see that there is an initial decreasing in the density.

\subsection{Community detection}
We finally want to analyze how the community detection varies along the coarse 
graining, and to understand if we are able to recover the given community through
the network structure using the Louvain algorithm, which optimizes the modularity
of the communities. We choose as resolution parameter $0.7$. It is 
linked to the precision in the community detection, i.e. the number 
of communities founded is inversely proportional to the resolution.
The modularity is defined as:
\begin{equation}
    Q = \frac{1}{2m}\sum_{ij} \left[A_{ij}-\frac{k_ik_j}{2m}\right] \delta(C_i, C_j)
\end{equation}
where $A_{ij}$ is the weighted adjacency matrix, $k_{i,j}$ is the weighted degree of node $i,j$,
$m$ is the sum of all the edge weights in the graph, $C_{i,j}$ are the communities $i,j$ and $\delta(x)$ is the
Kronecker delta function.

We take as correct the neuronal community defined in the previous section from the synapses,
$C_i^{(cor)}$, and compute the error $\epsilon_i$ of the community $C_i^{(alg)}$, computed by
the algorithm, as:
\begin{equation} \label{eq:com_er}
    \epsilon_i = 1- \frac{\left|C_i^{(cor)} \bigcap C_i^{(alg)}\right| }{\max{\left(\left|C_i^{(cor)}\right|, \left|C_i^{(alg)}\right|\right)}}
\end{equation}
where with $|\cdot|$ we denote the cardinality, i.e. the number of element, of a set and with $\bigcap$
the intersection operation. Basically we are looking at the number of shared neurons in the two communities, 
and then we divide by the largest community. In this way we do not incur in the problem 
of a big community which fully contains a smaller one having a small error. 
Notice that the error is defined in $[0,1]$.

We apply the Louvain algorithm both starting from the correct communities and without previous knowledge.
The results are shown in Figure \ref{fig:cmaps}. We present the results
using confusion matrices, where on the x-axis we have the correct labels
and on the y-axis the predicted ones. Notice that we do not have a preferred
order in the predicted regions, and it is so important to check the error with
all the possible regions. We notice that the two regions more easily
detected are the central complex and the mushroom body, which are well
detected also in the case of no previous knowledge. We notice that it is 
more difficult to detect the small communities, but it can be a bias of
how we have defined the error. We indeed notice a very interesting feature: 
with the coarse graining we decrease the error with no previous knowledge. For example, 
in the starting case we have an error on the central complex of $0.15$, which
becomes $0.14$ with $1700$ neurons and $0.06$ with $12000$ neurons! The error
increase again if we grain further, with $0.38$ with $7000$. Using the correct 
communities as starting communities we see that the MB and the CX are almost
perfectly detected. The others communities are instead not detected 
particularly well. We only have a partial recognition of VLNP.

What is important in this analysis is that we have highlight two regions
of the drosophila brain that are easily detected and modular. This result
is also interesting for the function of these two regions: they are both
involved in learning and memory. This detail can be interesting for further
studying. 