In this work we will analyze the structural connectome of a Drosophila, commonly known as 
“small fruit fly”. We will use the hemibrane dataset provided by the FlyEM project\cite{xu2020connectome}, which is composed by $21733$ neurons connected by $2760725$ synapses. The size
of the resulting network is overwhelming, and lots of the more advanced analysis are 
not feasible with this size. For this reason, borrowing the terminology from statistical
physics, we will apply a coarse graining to the network, reducing its size and monitoring
important metrics. 

The dataset contains a complete set of the neuron cell type, which we will
not use. Furthermore, it presents the synaptic weights and their brain zones. 
We will not go into the details of how this dataset has been constructed, since the understanding
of the biomedical techniques involved is beyond the scope of this report.
We will mainly use the python library Networkx\cite{hagberg2008exploring} for the analysis, 
which provides many useful and optimized algorithms for handling networks. All the code developed
for this project will be freely accessible on this github page \textbf{[INSERT HYPERLINK]}


This report will be organized as follows:
\begin{itemize}
    \item In Section \ref{sec:intro} we present this brief introduction;
    \item In Section \ref{sec:coarse} we present a definition of the coarse graining technique that we will use to reduce the size 
        of the network, with the evaluation of important metrics of the graph and their 
        evolution along the procedure;
    \item In Section \ref{sec:rob} we present the study of the network robustness, simulating brain damage. 
    \item In Section \ref{sec:comm} we present an application of the community detection algorithm, confronting the results
        with the knowledge provided by the dataset. In particular, we will try to understand
        also from a physiological point of view the results;
    \item In Section \ref{sec:concl} we conclude the work, wrapping up the results obtained, presenting some 
        ideas for future works;
    \item In Section \ref{sec:app} we provide an appendix, where we show all the figures
     of this report.
\end{itemize}
