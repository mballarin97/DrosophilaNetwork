In this work we will analyze the structural connectome of a Drosophila, commonly known as 
“small fruit fly”. We will use the hemibrane dataset provided by the FlyEM project\cite{xu2020connectome}, which is composed by $21733$ neurons connected by $2760725$ synapses. The size
of the resulting network is overwhelming, and lots of the more advanced analysis are 
not feasible in this framework. For this reason, borrowing the terminology from statistical
physics, we will apply a coarse graining to the network, reducing its size and monitoring
important metrics. 

The dataset contains a complete set of the neuron cell type, which we will
not use, and of the synaptic weights and their brain zones. We will mainly use
the python library Networkx\cite{hagberg2008exploring} for the analysis.


This report will be organized as follows:
\begin{enumerate}
    \item This brief introduction;
    \item A definition of the coarse graining technique that we will use to reduce the size 
        of the network, with the evaluation of important metrics of a graph. We will 
        focus on the results of the coarse graining on the network;
    \item The robustness of the network, simulating brain damage. 
    \item An application of the community detection algorithm, confronting the results
        with the knowledge that we have in the dataset. In particular, we will try to understand
        also from a physiological point of view which are the reasons of the results
        that we will obtain;
    \item A conclusion to wrap up the results obtained, presenting some 
        ideas for future works.
\end{enumerate}
