In this work we will analyze the structural connectome of a Drosophila, commonly known as 
“small fruit fly”. We will use the hemibrane dataset provided by the FlyEM project[instert
citation], which is composed by $20000$ neurons connected by $200000$ synapses. The size
of the resulting network is overwhelming, and lots of the more advanced analysis are 
not feasible in this framework. For this reason, borrowing the terminology from statistical
physics, we will apply a coarse graining to the network, reducing its size. We will then 
apply several different analysis. This report will be organized as follows:
\begin{enumerate}
    \item An introduction on the data available, focusing on the macro-areas of the Drosophila
        brain. We strongly believe that, even if a deep physiological understanding is not 
        necessary, it will really enhance the understanding of the measures we will apply,
        making this work not a simple application of the network science tools to a different
        topic, but an interesting case of study;
    \item A definition of the coarse graining technique that we will use to reduce the size 
        of the network;
    \item An analysis of the simplest network characteristics, on both the grained and the 
        original network, focusing on the evolution of such quantities;
    \item A research for the different communities in the networks;
    \item The robustness of the network, simulating brain damage. What happens if we damage
        the reduced network and then expand it to the original size? Is this equivalent to 
        remove the same number of nodes from the original network?
    \item A conclusion to wrap up our results.
\end{enumerate}

\red{\tb{The introduction is not finished yet. However, it is difficult to write it before
the end of the work.}}